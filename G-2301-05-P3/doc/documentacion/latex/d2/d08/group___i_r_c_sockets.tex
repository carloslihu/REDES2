\hypertarget{group___i_r_c_sockets}{\section{sockets}
\label{group___i_r_c_sockets}\index{sockets@{sockets}}
}
Funciones que se encargan de manejar sockets



 \hypertarget{openSocket_TCP}{}\subsubsection{open\-Socket\-\_\-\-T\-C\-P}\label{openSocket_TCP}
abre un socket T\-C\-P

\begin{DoxyReturn}{Returns}
el socket o -\/1 en caso de que hubiera un fallo
\end{DoxyReturn}


 \hypertarget{bindSocket_TCP}{}\subsubsection{bind\-Socket\-\_\-\-T\-C\-P}\label{bindSocket_TCP}
enlaza un socket T\-C\-P con un puerto a la vez que rellena los campos de la estructura sockaddr\-\_\-in de serv\-\_\-addr


\begin{DoxyParams}{Parameters}
{\em sockfd} & el socket a enlazar \\
\hline
{\em portno} & el puerto al que enlazar el socket \\
\hline
{\em serv\-\_\-addr} & la dirección de una estructura sockaddr\-\_\-in en donde se guardarán los valores relativos a esta conexión\\
\hline
\end{DoxyParams}
\begin{DoxyReturn}{Returns}
-\/1 en caso de error, un numero positivo en otro caso
\end{DoxyReturn}


 \hypertarget{acceptConnection}{}\subsubsection{accept\-Connection}\label{acceptConnection}
prepara un socket para aceptar una conexión entrante


\begin{DoxyParams}{Parameters}
{\em sockfd} & el socket que está escuchando conexiones entrantes \\
\hline
{\em cli\-\_\-addr} & la dirección de una estructura sockaddr\-\_\-in donde se rellenaran los valores relativos al otro extremo de la conexión\\
\hline
\end{DoxyParams}
\begin{DoxyReturn}{Returns}
-\/1 en caso de error y un nuevo socket (valor mayor que 0) en el que se mantendrá la comunicación con el otro extremo
\end{DoxyReturn}


 \hypertarget{connectTo}{}\subsubsection{connect\-To}\label{connectTo}
trata de conectarse a un puerto de un host


\begin{DoxyParams}{Parameters}
{\em sockfd} & el socket que intentará realizar la conexión \\
\hline
{\em hostname} & el nombre de la maquina a la que se desea conectarse \\
\hline
{\em portno} & el puerto al que se desea conectar\\
\hline
\end{DoxyParams}
\begin{DoxyReturn}{Returns}
en caso de error, un valor negativo. De lo contrario, un nuevo socket en el que se mantendrá la comunicación con el otro extremo
\end{DoxyReturn}


 \hypertarget{connectToIP}{}\subsubsection{connect\-To\-I\-P}\label{connectToIP}
trata de conectarse a un puerto de una I\-P


\begin{DoxyParams}{Parameters}
{\em sockfd} & el socket que intentará realizar la conexión \\
\hline
{\em I\-P} & la dirección I\-P en notación estándar a la que conectarse \\
\hline
{\em portno} & el puerto al que se desea conectar\\
\hline
\end{DoxyParams}
\begin{DoxyReturn}{Returns}
en caso de error, un valor negativo. De lo contrario, un nuevo socket en el que se mantendrá la comunicación con el otro extremo
\end{DoxyReturn}


 \hypertarget{openSocket_UDP}{}\subsubsection{open\-Socket\-\_\-\-U\-D\-P}\label{openSocket_UDP}
abre un socket U\-D\-P

\begin{DoxyReturn}{Returns}
el socket o -\/1 en caso de que hubiera un fallo
\end{DoxyReturn}


 \hypertarget{bindSocket_UDP}{}\subsubsection{bind\-Socket\-\_\-\-U\-D\-P}\label{bindSocket_UDP}
enlaza un socket U\-D\-P con un puerto a la vez que rellena los campos de la estructura sockaddr\-\_\-in de serv\-\_\-addr


\begin{DoxyParams}{Parameters}
{\em sockfd} & el socket a enlazar \\
\hline
{\em portno} & el puerto al que enlazar el socket \\
\hline
{\em serv\-\_\-addr} & la dirección de una estructura sockaddr\-\_\-in en donde se guardarán los valores relativos a esta conexión\\
\hline
\end{DoxyParams}
\begin{DoxyReturn}{Returns}
-\/1 en caso de error, un numero positivo en otro caso
\end{DoxyReturn}


 \hypertarget{getSocketPort}{}\subsubsection{get\-Socket\-Port}\label{getSocketPort}
obtiene el puerto al que ha sido enlazado el socket sockfd y tambien rellena los campos de serv


\begin{DoxyParams}{Parameters}
{\em sockfd} & el socket (enlazado) del que queremos sacar su puerto \\
\hline
{\em serv} & la estructura sockaddr\-\_\-in en la que se guardan los datos referentes a nuestro extremo de la conexion\\
\hline
\end{DoxyParams}
\begin{DoxyReturn}{Returns}
el puerto del socket. Si hubo un error, -\/1
\end{DoxyReturn}


 \hypertarget{iniAddrUDP}{}\subsubsection{ini\-Addr\-U\-D\-P}\label{iniAddrUDP}
inicializa correctamente la estructura con datos para una conexion U\-D\-P


\begin{DoxyParams}{Parameters}
{\em si\-\_\-other} & la estructura que se inicializara \\
\hline
{\em port} & el puerto del otro sistema con el que se desea comunicar \\
\hline
{\em hostname} & el hostname del otro sistema con el que se desea comunicar\\
\hline
\end{DoxyParams}
\begin{DoxyReturn}{Returns}
-\/1 si hubo un error, I\-R\-C\-\_\-\-O\-K si todo fue bien
\end{DoxyReturn}


 